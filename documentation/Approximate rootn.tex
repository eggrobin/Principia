\documentclass[10pt, a4paper, twoside]{basestyle}
\usepackage{tkz-euclide}
\usetkzobj{all}
\usepackage[Mathematics]{semtex}
\usepackage{chngcntr}
\counterwithout{equation}{section}

%%%% Shorthands.
\newcommand{\idest}{\emph{, i.e.,\ }}

%%%% Title and authors.

\newcommand{\point}[1]{\mathrm{#1}}
\newcommand{\bipoint}[2]{\overrightarrow{\point #1 \point #2}}
\newcommand{\straightline}[2]{\point #1 \point #2}
\newcommand{\plane}[3]{\point #1 \point #2 \point #3}
\newcommand{\squarenorm}[1]{\scal{#1}{#1}}

\newcommand{\ApproximateRoot}[1]{{{}^{#1}r}}

\DeclareDelimiter\reversedceil{\rceil}{\lceil}
\newcommand\ModOne[1]{\reversedceil{#1}}

\renewcommand\labelitemi{---}

\title{%
\textdisplay{%
Approximating roots and reciprocal roots of binary floating-point numbers%
}%
}
\author{Robin~Leroy (eggrobin)}
\begin{document}
\maketitle
In the following, $\Naturals\DefineAs\intclop{0}{\infty}\Intersection\Z$.
We define $\ModOne{x}\DefineAs x - \Floor{x}$, so that $\forall x\in\R, \ModOne{x}\in\intclop 0 1$.
$B\in\Naturals$ is arbitrary.

Let $x>0$. There are unique $F\in\intclop 0 1$, $K\in\Integers$, such that $x=2^K\pa{1+F}$;
define\[
x\sharp \DefineAs B+K+F\text.
\]
Let $X\in\Reals$; define\[
X\flat \DefineAs 2^{\Floor{X-B}} \pa{ 1 + \ModOne{X} }\text.
\]
Then $X\flat\sharp = X$, $x\sharp\flat=x$, $x\sharp+1 = \pa{2x}\sharp$.

Let $n\in\Z\setminus\set{0,1}$, $\gg\in\R$. For $x>0$, define\[
\ApproximateRoot{n}\of{x} \DefineAs
\pa{C_{n,\gg} + \frac{x\sharp}{n}}\flat\text,
\]
where
\[
C_{n,\gg} \DefineAs \frac{\pa{n-1}B-\gg}{n}\text.
\]
Consider the signed relative error $\ge\of x$ of $\ApproximateRoot{n}\of{x}$ as an approximation of
$\sqrt[n]{x}$.
For $x=2^K\pa{1+F}$, we have
\begin{align*}
\ge\of x &= \frac{\ApproximateRoot{n}\of{x}}{\sqrt[n]{x}} - 1 \\
&= 
\frac{
  2^{\Floor{\frac{K+F-\gg}{n}}} \pa{1+\ModOne{\frac{K+F-\gg}{n}}}
}
{
  2^{\frac{K}{n}} \sqrt[n]{1+F}
} - 1 \\
&= 2^{\Floor{\frac{K+F-\gg}{n}} - \frac{K}{n}} \frac{1+\ModOne{\frac{K+F-\gg}{n}}}{\sqrt[n]{1+F}} - 1 \text,
\end{align*}
which is invariant under addition of $n$ to $K$, so that\[
\ge\of{x} = \ge\of{2^nx}\text.
\]
in other words,\[
\FunctionNamedBody{\ge_\flat}{X}{\ge\of{X\flat}}
\]
is periodic with period $n$.

Consider the interval \[
I_{n,\gg}\DefineAs \begin{cases}
\intclop{2^{\Floor{\gg}}\pa{1+\ModOne{\gg}}} {2^{\Floor{\gg}+n}\pa{1+\ModOne{\gg}}} & n>0\text,\\
\intclop{2^{\Floor{\gg}+n}\pa{1+\ModOne{\gg}}} {2^{\Floor{\gg}}\pa{1+\ModOne{\gg}}} & \text{otherwise.}
\end{cases}
\]
Note that \[
I_{n,\gg}\sharp =\begin{cases}
\intclop{B+\gg} {B+n+\gg} & n>0\text,\\
\intclop{B+n+\gg} {B+\gg} & \text{otherwise,}
\end{cases}
\]
so that it covers one period of the relative error.

Let $x\in I_{n,\gg}$.
Then, with $F\in\intclop 0 1$, $K\in\Integers$, such that $x=2^K\pa{1+F}$,
\[
\ApproximateRoot{n}\of{x} = 1+\frac{K+F-\gg}{n} = 1+\frac{K+2^{-K}x-1-\gg}{n}\text,
\]
and $K\in \intclos{\Floor{\gg}}{\Floor{\gg}+n-1}\Intersection\Z$ if $n>0$,
$K\in \intclos{\Floor{\gg}+n}{\Floor{\gg}}\Intersection\Z$ otherwise.
For fixed $K$\idest for $x\in\intclop{2^K}{2^{K+1}}$,
$\ge\der\of{x}=0$ at \[
x=2^K\pa{1+\frac{K-\gg}{n-1}}\text,
\]
which is in $\intclop{2^K}{2^{K+1}}$ unless $K=\Floor{\gg}$ and $n>0$, or $K=\Floor{\gg}+n$ and $n>0$.

It follows that the maximum for $x>0$ of $\abs{\ge\of{x}}$ is the maximum of the absolute values of the
following:
\begin{itemize}
\item the value $\ge\of{2^{\Floor{\gg}}\pa{1+\ModOne{\gg}}}=
\frac{1}{\sqrt[n]{2^{\Floor{\gg}}\pa{1+\ModOne{\gg}}}}-1$ at the endpoint of $I_{n,\gg}$;
\item the values at powers of two within $I_{n,\gg}$,
$\ge\of{2^K}=2^{-\frac{K}{n}}\pa{1+\frac{K-\gg}{n}}-1$ for
$K\in \intclos{\Floor{\gg}}{\Floor{\gg}+n-1}\Intersection\Z$ if $n>0$,
$K\in \intclos{\Floor{\gg}+n}{\Floor{\gg}}\Intersection\Z$ otherwise;
\item the smooth extrema, $\ge\of{2^K\pa{1+\frac{K-\gg}{n-1}}}$ where
$K\in \intclos{\Floor{\gg}+1}{\Floor{\gg}+n-2}\Intersection\Z$ if $n>0$ and
$K\in \intclos{\Floor{\gg}+n}{\Floor{\gg}}\Intersection\Z$ otherwise.
\end{itemize}
\end{document}

\documentclass[10pt, a4paper, twoside]{basestyle}

\usepackage{tikz}
\usetikzlibrary{cd}

\usepackage[Mathematics]{semtex}

%%%% Shorthands.

%%%% Title and authors.

\title{Celestial double ledger accounting}
\date{\printdate{2023-04-08}}
\author{Lamont~Granquist and Robin~Leroy (egg)}
\begin{document}
\maketitle

\section{Introduction}\label{introduction}
Consider a rocket with $9\,000\text{ m/s}$ of vacuum $\increment v$, \idest,
capable of reaching $9\,000\text{ m/s}$ from a standstill in a vacuum.
Launch it from Earth; it starts out going at $400\text{ m/s}$ eastward, yet
it runs out of fuel in orbit going $7\,800\text{ m/s}$.
What happened to $1\,400\text{ m/s}$?
Consider another rocket with $12\,000\text{ m/s}$ of vacuum $\increment v$,
launch it from Earth, it runs out of fuel reaching the same orbit as the other one.
What happened to another $3\,000\text{ m/s}$?

The vacuum $\increment v$ of a rocket is
\begin{equation}
\increment v_{\text{vac}}=
\log\frac{m_{\text{final}}}{m_{\text{initial}}}
I_{\text{sp vac}}
=
\int{m_{\text{initial}}}[m_{\text{final}}]
\frac{I_{\text{sp vac}}}{m} \diffd m \text,\label{vacuumΔv}
\end{equation}
where we use specific impulse \emph{stricto sensu},
impulse divided by mass\footnote{For specific
impulse with a dimension of time, multiply it by standard gravity to
give it the dimension of impulse divided by mass;
we do not involve standard gravity in the equations here
as it would only serve to obscure the physics.},
whose SI unit is the Newton-second per kilogram.


\section{Losses}
Consider a rocket of time-varying mass $m$ with velocity $\vv$, subject to 
an overall force $\vF$.
The rate of change of its speed is
\begin{equation}
\deriv t v = \derivop t \sqrt{\scal{\vv}{\vv}}
= \frac{\scal{\TimeDerivative\vv}{\vv}}{\sqrt{\scal{\vv}{\vv}}}
= \scal{\TimeDerivative\vv}{\hat{\vv}}
= \scal{\frac{\vF}{m}}{\hat{\vv}}\text.\label{speedDerivative}
\end{equation}
Splitting up the force $\vF$, working in an inertial reference frame
so that we do not get fictitious forces, we get
\begin{equation}
\deriv t v =
\scal{\frac{\vF_{\text{thrust}}}{m}}{\hat{\vv}} +
\scal{\frac{\vF_{\text{aerodynamic}}}{m}}{\hat{\vv}} +
\scal{\frac{\vF_{\text{gravitational}}}{m}}{\hat{\vv}}\text.
\label{forces}
\end{equation}
Aerodynamic forces along the direction of travel have a name,
and they really go \emph{against} the direction of travel:
\[
\deriv t v =
\scal{\frac{\vF_{\text{thrust}}}{m}}{\hat{\vv}} -
\frac{F_{\text{drag}}}{m} +
\scal{\frac{\vF_{\text{gravitational}}}{m}}{\hat{\vv}}\text.
\]
Only the thrust along the direction of travel
contributes to the increase in velocity:
\[
\deriv t v =
\frac{F_{\text{thrust}}}{m} -
\frac{F_{\text{thrust}}}{m}
\pa{1-\scal{\hat{\vF}_{\text{thrust}}}{\hat{\vv}}} -
\frac{F_{\text{drag}}}{m} +
\scal{\frac{\vF_{\text{gravitational}}}{m}}{\hat{\vv}}\text.
\]
The thrust of a rocket depends on the current specific impulse and mass flow:
\[
\deriv t v =
\frac{I_{\text{sp}}}{m} \TimeDerivative m -
\frac{F_{\text{thrust}}}{m}
\pa{1-\scal{\hat{\vF}_{\text{thrust}}}{\hat{\vv}}} -
\frac{F_{\text{drag}}}{m} +
\scal{\frac{\vF_{\text{gravitational}}}{m}}{\hat{\vv}}\text.
\]
In turn the specific impulse diminishes with atmospheric pressure:
\[
\deriv t v =
\frac{I_{\text{sp vac}}}{m} \TimeDerivative m -
\frac{\increment I_{\text{sp}}\of{P}}	{m} \TimeDerivative m -
\frac{F_{\text{thrust}}}{m}
\pa{1-\scal{\hat{\vF}_{\text{thrust}}}{\hat{\vv}}} -
\frac{F_{\text{drag}}}{m} +
\scal{\frac{\vF_{\text{gravitational}}}{m}}{\hat{\vv}}\text.
\label{instantaneouslosses}
\]
Compare with (\ref{vacuumΔv}) and observe that the first term here is the rate of consumption of vacuum $\increment v$.
Integrating over the ascent, we get the relation between the actual change in speed
and the expended vacuum $\increment v$:
\begin{align}
\increment v_{\text{actual}} =
\increment v_{\text{vac}} &-
\underbrace{
\int{}
\increment I_{\text{sp}}\of{P} \frac{\diffd m}{m}
}_{\text{backpressure losses}} -
\underbrace{
\int{}
\frac{F_{\text{thrust}}}{m}
\pa{1-\scal{\hat{\vF}_{\text{thrust}}}{\hat{\vv}}}\diffd t
}_{\text{steering losses}}\nonumber
 \\&-
\underbrace{
\int{}
\frac{F_{\text{drag}}}{m}\diffd t
}_{\text{drag losses}} +
\underbrace{
\int{}
\scal{\frac{\vF_{\text{gravitational}}}{m}}{\hat{\vv}}\diffd t
}_{\text{gravity losses}}\text.\label{losses}
\end{align}
The terms that account for the difference between the actual change in speed
and the expended vacuum $\increment v$ are termed \emph{losses}.
The names \emph{steering}, \emph{drag}, and \emph{gravity} losses are from [TODO CITE].

Backpressure and drag losses are reasonably self-explanatory;
in order to better understand the other two terms, we will consider the case where they vanish,
\idest, a vacuum, in the remainder of this document:
\begin{equation}
\increment v_{\text{actual}} =
\increment v_{\text{vac}} -
\underbrace{
\int{t_{\text{initial}}}[t_{\text{final}}]
\frac{F_{\text{thrust}}}{m}
\pa{1-\scal{\hat{\vF}_{\text{thrust}}}{\hat{\vv}}}\diffd t
}_{\text{steering losses}}
+
\underbrace{
\int{t_{\text{initial}}}[t_{\text{final}}]
\scal{\frac{\vF_{\text{gravitational}}}{m}}{\hat{\vv}}\diffd t
}_{\text{gravity losses}}\text.\label{lossesVacuum}
\end{equation}

\section{Work and Oberth}

While equation (\ref{losses}) accounts for the difference described in Section \ref{introduction},
not all terms represent actual losses. In particular, in free fall, gravity losses accrue
when going up, but are recouped when going down; they cycle over the course of a revolution 
on an eccentric orbit.

Consider the specific kinetic energy $T=\frac{v^2}{2}$. Its rate of change is\begin{equation}
\TimeDerivative T = \deriv t v v = \scal{\frac{\vF}{m}}{\vv}\text,\label{kineticEnergyDerivative}
\end{equation}
the specific work done by $\vF$, where we have used (\ref{speedDerivative}).
Thus each term in (\ref{forces}) corresponds to the work done by one of the forces.
In particular, steering losses correspond to thrust not doing work, \idest, not contributing to
the orbital energy.

The interpretation of gravity losses is more complex;
gravity preserves the sum of the specific kinetic energy and the
specific gravitational potential energy:
work done by gravity is gained as potential energy.
We could split and integrate (\ref{kineticEnergyDerivative}) like (\ref{speedDerivative}) to get\begin{equation}
\increment T =
\underbrace{
\int{}\frac{F_{\text{thrust}}}{m}\diffd s
}_{E_{\text{thrust}}} -
\underbrace{
\int{}\frac{F_{\text{thrust}}}{m}\pa{1-\scal{\hat{\vF}_{\text{thrust}}}{\diffd \vs}}
}_{E_{\text{thrust}}-W_{\text{thrust}}} +
\increment V_{\text{gravitational}}\text.
\label{energies}
\end{equation}
There is however an important difference between equations (\ref{energies}) and (\ref{lossesVacuum}).
The energy $E_{\text{thrust}}$ is not a fixed property of the rocket like the $\increment v$ budget;
it is path dependent, as the same $\increment v$ applied at greater speed results in a greater gain
in kinetic energy; thus in order to maximize the gain in orbital energy, beyond thrusting along
the direction of travel, one needs to maximize the speed at which thrust is applied: this is the Oberth effect [TODO CITE Oberth Ch. 12].

Conversely, whereas $\increment V_{\text{gravitational}}$ is path-independent in (\ref{energies}),
gravity losses are path-dependent, \idest, they describe some property of the ascent profile.
Specifically, gravity losses are lessened if the rocket has a greater speed (in any direction) when
rising in the potential.
In turn this means that this speed should have been gained lower in the potential, thus
while going faster: this is a roundabout expression of the Oberth effect.
\paragraph{Summary:}\begin{enumerate}
\item steering losses are $\increment v$ expended not increasing the orbital energy;
\item gravity losses are a combination of:\begin{enumerate}
\item $\increment v$ turned into
gravitational potential energy (thus they can never be $0$, even with a horizontal
instant impulse, so long as the orbit is above the ground),
\item an expression of the Oberth effect (so they depend on the ascent profile).
\end{enumerate}
\end{enumerate}

\section{Thrust up bad, thrust sideways good}
KSP players have a strong intuition that thrust expended upward is wasted.

There are cases where upward thrust is related to the losses described in preceding section;
for instance, at launch on a rotating planet, thrust upward is pure steering losses, being
orthogonal to the direction of motion.
With low thrust, thrusting upward also causes the rocket to rise in the potential with a
lower speed than thrusting sideways, so it causes gravity losses; but this is a fairly
high order effect.
Indeed, if one considers a launch with instant impulse from a non-rotating planet,
there are no steering losses and the impulse takes full advantage of the Oberth effect;
yet in a sense that impulse is still wasted,
in that it will not help the vessel get into orbit.

The intuition that thrust should be sideways is instead related to the other invariant of the orbit,
angular momentum: only horizontal thrust contributes to it.
An accounting of horizontal speed similar to (\ref{losses}) may be used to express that.

[TODO(egg): do it.]

\end{document}